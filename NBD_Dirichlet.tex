% Options for packages loaded elsewhere
% Options for packages loaded elsewhere
\PassOptionsToPackage{unicode}{hyperref}
\PassOptionsToPackage{hyphens}{url}
\PassOptionsToPackage{dvipsnames,svgnames,x11names}{xcolor}
%
\documentclass[
  a4paper,
  DIV=11,
  numbers=noendperiod]{scrartcl}
\usepackage{xcolor}
\usepackage{amsmath,amssymb}
\setcounter{secnumdepth}{5}
\usepackage{iftex}
\ifPDFTeX
  \usepackage[T1]{fontenc}
  \usepackage[utf8]{inputenc}
  \usepackage{textcomp} % provide euro and other symbols
\else % if luatex or xetex
  \usepackage{unicode-math} % this also loads fontspec
  \defaultfontfeatures{Scale=MatchLowercase}
  \defaultfontfeatures[\rmfamily]{Ligatures=TeX,Scale=1}
\fi
\usepackage{lmodern}
\ifPDFTeX\else
  % xetex/luatex font selection
\fi
% Use upquote if available, for straight quotes in verbatim environments
\IfFileExists{upquote.sty}{\usepackage{upquote}}{}
\IfFileExists{microtype.sty}{% use microtype if available
  \usepackage[]{microtype}
  \UseMicrotypeSet[protrusion]{basicmath} % disable protrusion for tt fonts
}{}
\makeatletter
\@ifundefined{KOMAClassName}{% if non-KOMA class
  \IfFileExists{parskip.sty}{%
    \usepackage{parskip}
  }{% else
    \setlength{\parindent}{0pt}
    \setlength{\parskip}{6pt plus 2pt minus 1pt}}
}{% if KOMA class
  \KOMAoptions{parskip=half}}
\makeatother
% Make \paragraph and \subparagraph free-standing
\makeatletter
\ifx\paragraph\undefined\else
  \let\oldparagraph\paragraph
  \renewcommand{\paragraph}{
    \@ifstar
      \xxxParagraphStar
      \xxxParagraphNoStar
  }
  \newcommand{\xxxParagraphStar}[1]{\oldparagraph*{#1}\mbox{}}
  \newcommand{\xxxParagraphNoStar}[1]{\oldparagraph{#1}\mbox{}}
\fi
\ifx\subparagraph\undefined\else
  \let\oldsubparagraph\subparagraph
  \renewcommand{\subparagraph}{
    \@ifstar
      \xxxSubParagraphStar
      \xxxSubParagraphNoStar
  }
  \newcommand{\xxxSubParagraphStar}[1]{\oldsubparagraph*{#1}\mbox{}}
  \newcommand{\xxxSubParagraphNoStar}[1]{\oldsubparagraph{#1}\mbox{}}
\fi
\makeatother


\usepackage{longtable,booktabs,array}
\usepackage{calc} % for calculating minipage widths
% Correct order of tables after \paragraph or \subparagraph
\usepackage{etoolbox}
\makeatletter
\patchcmd\longtable{\par}{\if@noskipsec\mbox{}\fi\par}{}{}
\makeatother
% Allow footnotes in longtable head/foot
\IfFileExists{footnotehyper.sty}{\usepackage{footnotehyper}}{\usepackage{footnote}}
\makesavenoteenv{longtable}
\usepackage{graphicx}
\makeatletter
\newsavebox\pandoc@box
\newcommand*\pandocbounded[1]{% scales image to fit in text height/width
  \sbox\pandoc@box{#1}%
  \Gscale@div\@tempa{\textheight}{\dimexpr\ht\pandoc@box+\dp\pandoc@box\relax}%
  \Gscale@div\@tempb{\linewidth}{\wd\pandoc@box}%
  \ifdim\@tempb\p@<\@tempa\p@\let\@tempa\@tempb\fi% select the smaller of both
  \ifdim\@tempa\p@<\p@\scalebox{\@tempa}{\usebox\pandoc@box}%
  \else\usebox{\pandoc@box}%
  \fi%
}
% Set default figure placement to htbp
\def\fps@figure{htbp}
\makeatother





\setlength{\emergencystretch}{3em} % prevent overfull lines

\providecommand{\tightlist}{%
  \setlength{\itemsep}{0pt}\setlength{\parskip}{0pt}}



 


\KOMAoption{captions}{tableheading}
\usepackage[none]{hyphenat}
\usepackage{ragged2e}
\RaggedRight
\makeatletter
\@ifpackageloaded{tcolorbox}{}{\usepackage[skins,breakable]{tcolorbox}}
\@ifpackageloaded{fontawesome5}{}{\usepackage{fontawesome5}}
\definecolor{quarto-callout-color}{HTML}{909090}
\definecolor{quarto-callout-note-color}{HTML}{0758E5}
\definecolor{quarto-callout-important-color}{HTML}{CC1914}
\definecolor{quarto-callout-warning-color}{HTML}{EB9113}
\definecolor{quarto-callout-tip-color}{HTML}{00A047}
\definecolor{quarto-callout-caution-color}{HTML}{FC5300}
\definecolor{quarto-callout-color-frame}{HTML}{acacac}
\definecolor{quarto-callout-note-color-frame}{HTML}{4582ec}
\definecolor{quarto-callout-important-color-frame}{HTML}{d9534f}
\definecolor{quarto-callout-warning-color-frame}{HTML}{f0ad4e}
\definecolor{quarto-callout-tip-color-frame}{HTML}{02b875}
\definecolor{quarto-callout-caution-color-frame}{HTML}{fd7e14}
\makeatother
\makeatletter
\@ifpackageloaded{caption}{}{\usepackage{caption}}
\AtBeginDocument{%
\ifdefined\contentsname
  \renewcommand*\contentsname{Table of contents}
\else
  \newcommand\contentsname{Table of contents}
\fi
\ifdefined\listfigurename
  \renewcommand*\listfigurename{List of Figures}
\else
  \newcommand\listfigurename{List of Figures}
\fi
\ifdefined\listtablename
  \renewcommand*\listtablename{List of Tables}
\else
  \newcommand\listtablename{List of Tables}
\fi
\ifdefined\figurename
  \renewcommand*\figurename{Figure}
\else
  \newcommand\figurename{Figure}
\fi
\ifdefined\tablename
  \renewcommand*\tablename{Table}
\else
  \newcommand\tablename{Table}
\fi
}
\@ifpackageloaded{float}{}{\usepackage{float}}
\floatstyle{ruled}
\@ifundefined{c@chapter}{\newfloat{codelisting}{h}{lop}}{\newfloat{codelisting}{h}{lop}[chapter]}
\floatname{codelisting}{Listing}
\newcommand*\listoflistings{\listof{codelisting}{List of Listings}}
\makeatother
\makeatletter
\makeatother
\makeatletter
\@ifpackageloaded{caption}{}{\usepackage{caption}}
\@ifpackageloaded{subcaption}{}{\usepackage{subcaption}}
\makeatother
\usepackage{bookmark}
\IfFileExists{xurl.sty}{\usepackage{xurl}}{} % add URL line breaks if available
\urlstyle{same}
\hypersetup{
  pdftitle={The Negative Binomial--Dirichlet Model},
  pdfauthor={Hume Winzar},
  colorlinks=true,
  linkcolor={blue},
  filecolor={Maroon},
  citecolor={Blue},
  urlcolor={Blue},
  pdfcreator={LaTeX via pandoc}}


\title{The Negative Binomial--Dirichlet Model}
\author{Hume Winzar}
\date{}
\begin{document}
\maketitle

\renewcommand*\contentsname{Table of contents}
{
\hypersetup{linkcolor=}
\setcounter{tocdepth}{3}
\tableofcontents
}

\section*{Introduction}\label{introduction}
\addcontentsline{toc}{section}{Introduction}

\nopagebreak

The \textbf{Negative Binomial Distribution (NBD)} and the
\textbf{Dirichlet distribution} together form the statistical foundation
of modern marketing science.\\
They underpin the analyses presented in \emph{How Brands Grow} (Sharp,
2010) and explain regular patterns such as \textbf{brand penetration},
\textbf{repeat purchasing}, and \textbf{Duplication of Purchase (DoP)}.

\nopagebreak

The combined model is known as the \textbf{NBD--Dirichlet model}.

\begin{center}\rule{0.5\linewidth}{0.5pt}\end{center}

\section{Purchase Incidence --- The Negative Binomial
Distribution}\label{purchase-incidence-the-negative-binomial-distribution}

\nopagebreak

The Negative Binomial Distribution models the \textbf{number of
purchases per person} in a given period for a whole \textbf{product
category}.

It assumes:\\
- Each individual has their own average purchase rate, given by mu, (
\(\mu_i\) ), or lambda, ( \(\lambda_i\) ),\\
- Across people, ( \(\lambda_i\) ) follows a \emph{Gamma distribution}
(\(\Gamma(\cdot)\)),\\
- Purchases for each individual follow a \emph{Poisson process}.

A \emph{Gamma distribution} ( \(\Gamma(\cdot)\) ) looks like this:

\begin{figure}[H]

{\centering \includegraphics[width=4.16667in,height=\textheight,keepaspectratio]{images/Gamma_distributions.png}

}

\caption{Gamma distributions with varying shape parameters}

\end{figure}%

A \emph{Poisson distribution} is a probability distribution that
describes the number of events that occur within a fixed interval of
time or space. If mu (\(\mu\)) is the mean occurrence per interval, then
the probability of having \(x\) occurrences within a given interval is:

\[
P(X=x) = \frac{e^{−\mu} \cdot \mu^{k}}{k!}
\]

\begin{itemize}
\item
  \(P(X=x)\) represents the probability of observing \(x\) events.
\item
  \(e\) is the base of the natural logarithm.
\item
  \(\mu\) \emph{(mu)} is the average rate of event occurrences in a
  fixed time period.
\item
  \(x\) is the actual number of events observed - integers obviously.
\item
  \(k!\) denotes the factorial of \(k\), which is the product of all
  positive integers up to \(k\). That is,
  (\(1 \times 2 \times 3 \times \dots \times k\)).
\end{itemize}

The Poisson density function looks like this:

\nopagebreak

\begin{figure}[H]

{\centering \includegraphics[width=4.16667in,height=\textheight,keepaspectratio]{images/poisson_distribution.png}

}

\caption{Poisson density distribution}

\end{figure}%

\nopagebreak

\emph{(We use line charts here for ease of viewing three Poisson
distributions: Ordinarily, we would use bar-charts to show integer
values on the x-axis. We can't have ``half an event'')}

Mixing the \emph{Gamma} and \emph{Poisson} distributions yields the
\emph{Negative Binomial}.

\subsection{Negative Binomial Formula}\label{negative-binomial-formula}

\nopagebreak

\[
P(X = x) =
\frac{\Gamma(x + k)}{\Gamma(k)\, x!}
\left(\frac{k}{k + \mu}\right)^{k}
\left(\frac{\mu}{k + \mu}\right)^{x},
\quad x = 0,1,2,\ldots
\]

where:

\nopagebreak

\begin{longtable}[]{@{}
  >{\raggedright\arraybackslash}p{(\linewidth - 2\tabcolsep) * \real{0.2500}}
  >{\raggedright\arraybackslash}p{(\linewidth - 2\tabcolsep) * \real{0.7500}}@{}}
\toprule\noalign{}
\begin{minipage}[b]{\linewidth}\raggedright
Symbol
\end{minipage} & \begin{minipage}[b]{\linewidth}\raggedright
Meaning
\end{minipage} \\
\midrule\noalign{}
\endhead
\bottomrule\noalign{}
\endlastfoot
\(X\) & Number of purchases per person per period \\
\(\mu\) & Mean purchase rate per person \\
\(k\) & Dispersion (heterogeneity) parameter \\
\(\Gamma(\cdot)\) & Gamma function \\
\end{longtable}

The \emph{mean} and \emph{variance} are:

\nopagebreak

\[
E[X] = \mu, \qquad Var[X] = \mu\left(1 + \frac{\mu}{k}\right) 
\]

\begin{center}\rule{0.5\linewidth}{0.5pt}\end{center}

\subsection{Zero-Class Probability}\label{zero-class-probability}

\nopagebreak

The probability that a randomly chosen person makes \emph{zero}
purchases is:

\nopagebreak

\[ 
P(0) = \left(\frac{k}{k+\mu}\right)^k 
\]

Hence, the \textbf{category penetration} is:

\nopagebreak

\[ 
\text{Penetration} = 1 - P(0) 
\]

\nopagebreak

\begin{longtable}[]{@{}
  >{\raggedright\arraybackslash}p{(\linewidth - 2\tabcolsep) * \real{0.2500}}
  >{\raggedright\arraybackslash}p{(\linewidth - 2\tabcolsep) * \real{0.7500}}@{}}
\toprule\noalign{}
\begin{minipage}[b]{\linewidth}\raggedright
Concept
\end{minipage} & \begin{minipage}[b]{\linewidth}\raggedright
Interpretation
\end{minipage} \\
\midrule\noalign{}
\endhead
\bottomrule\noalign{}
\endlastfoot
\(P(0)\) & Proportion of people who make no purchases \\
\(1 - P(0)\) & Proportion of people who make at least one purchase \\
\(k\) & Degree of heterogeneity --- smaller (\(k\)) means greater
variability \\
\end{longtable}

\begin{center}\rule{0.5\linewidth}{0.5pt}\end{center}

\section{Brand Choice --- The Dirichlet
Component}\label{brand-choice-the-dirichlet-component}

\nopagebreak

Once we know how often people buy the category, we can model \emph{which
brand} they buy.

The \textbf{Dirichlet distribution} models how individuals allocate
their purchases across brands.

\begin{longtable}[]{@{}
  >{\raggedright\arraybackslash}p{(\linewidth - 2\tabcolsep) * \real{0.2500}}
  >{\raggedright\arraybackslash}p{(\linewidth - 2\tabcolsep) * \real{0.7500}}@{}}
\toprule\noalign{}
\begin{minipage}[b]{\linewidth}\raggedright
Symbol
\end{minipage} & \begin{minipage}[b]{\linewidth}\raggedright
Meaning
\end{minipage} \\
\midrule\noalign{}
\endhead
\bottomrule\noalign{}
\endlastfoot
\(s_i\) & Market share of brand `i' \\
\(m\) & Number of competing brands \\
\(\alpha_i\) & Dirichlet parameter for brand `i' \\
\end{longtable}

Each consumer has a personal probability vector
(\(p_1, p_2, \dots, p_m\)) representing their purchase probabilities
across brands,\\
and across consumers these follow a Dirichlet distribution:

\nopagebreak

\[ 
(p_1, \dots, p_m)  \sim  \text{Dirichlet}(\alpha_1, \dots, \alpha_m)
\]

Combining the NBD (purchase incidence) with the Dirichlet (brand choice)
gives the \textbf{NBD--Dirichlet model}.

This combined model predicts:

\nopagebreak

\begin{itemize}
\tightlist
\item
  \textbf{Brand penetration} (how many people buy each brand),
\item
  \textbf{Purchase frequency per buyer},
\item
  \textbf{Duplication of purchase} (brand overlap),
\item
  \textbf{Market share} and \textbf{loyalty patterns}.
\end{itemize}

\begin{center}\rule{0.5\linewidth}{0.5pt}\end{center}

\section{How the Two Components Fit
Together}\label{how-the-two-components-fit-together}

\nopagebreak

\begin{longtable}[]{@{}
  >{\raggedright\arraybackslash}p{(\linewidth - 4\tabcolsep) * \real{0.2500}}
  >{\raggedright\arraybackslash}p{(\linewidth - 4\tabcolsep) * \real{0.2500}}
  >{\raggedright\arraybackslash}p{(\linewidth - 4\tabcolsep) * \real{0.5000}}@{}}
\toprule\noalign{}
\begin{minipage}[b]{\linewidth}\raggedright
Concept
\end{minipage} & \begin{minipage}[b]{\linewidth}\raggedright
Source
\end{minipage} & \begin{minipage}[b]{\linewidth}\raggedright
Description
\end{minipage} \\
\midrule\noalign{}
\endhead
\bottomrule\noalign{}
\endlastfoot
\(\mu\) & NBD & Mean category purchase rate per person \\
\(k\) & NBD & Dispersion (heterogeneity) parameter \\
\(P(0)\) & NBD & Probability of zero category purchases \\
\(1 - P(0)\) & NBD & Category penetration \\
\(s_i\) & Dirichlet & Market share of brand \('i'\) \\
\(\alpha_i\) & Dirichlet & Shape parameter proportional to brand
share \\
Brand penetration & NBD--Dirichlet & \% of people who buy brand \('i'\)
at least once \\
Duplication of purchase & NBD--Dirichlet & \% of buyers of Brand A who
also buy Brand B \\
\end{longtable}

\begin{center}\rule{0.5\linewidth}{0.5pt}\end{center}

\subsection{Visual Summary}\label{visual-summary}

\nopagebreak

\includegraphics[width=2.88in,height=6.73in]{NBD_Dirichlet_files/figure-latex/mermaid-figure-1.png}

\section{Interpretation and Marketing
Insights}\label{interpretation-and-marketing-insights}

\nopagebreak

\begin{quote}
``Growth comes from increasing \textbf{penetration}, not from chasing
loyalty.''\\
--- \emph{How Brands Grow}, Byron Sharp (2010)
\end{quote}

The NBD--Dirichlet model demonstrates that many brand performance
patterns are \textbf{statistical regularities} rather than unique

Key insights:

\nopagebreak

\begin{itemize}
\tightlist
\item
  \textbf{Big brands} have \emph{more buyers} (low \(P(0)\)) and
  slightly higher repeat-purchase rates.
\item
  \textbf{Small brands} have \emph{fewer buyers} (high \(P(0)\)) and
  slightly lower repeat-purchase rates.
\item
  Buyer behaviour differences largely reflect \textbf{market share and
  penetration}, not deep attitudinal differences.
\end{itemize}

These principles explain two major empirical ``laws'' of marketing:

\begin{center}\rule{0.5\linewidth}{0.5pt}\end{center}

\subsection{Double Jeopardy}\label{double-jeopardy}

\nopagebreak

Small brands suffer \emph{twice}: they have \textbf{fewer buyers} (low
penetration) and those buyers are \textbf{slightly less loyal}.

\begin{longtable}[]{@{}
  >{\raggedright\arraybackslash}p{(\linewidth - 6\tabcolsep) * \real{0.1644}}
  >{\raggedright\arraybackslash}p{(\linewidth - 6\tabcolsep) * \real{0.1781}}
  >{\raggedright\arraybackslash}p{(\linewidth - 6\tabcolsep) * \real{0.4384}}
  >{\raggedright\arraybackslash}p{(\linewidth - 6\tabcolsep) * \real{0.2192}}@{}}
\toprule\noalign{}
\begin{minipage}[b]{\linewidth}\raggedright
Brand Type
\end{minipage} & \begin{minipage}[b]{\linewidth}\raggedright
Penetration
\end{minipage} & \begin{minipage}[b]{\linewidth}\raggedright
Purchase Frequency (per buyer)
\end{minipage} & \begin{minipage}[b]{\linewidth}\raggedright
Illustration
\end{minipage} \\
\midrule\noalign{}
\endhead
\bottomrule\noalign{}
\endlastfoot
Large & High & Slightly higher & e.g.~Coca-Cola \\
Small & Low & Slightly lower & e.g.~Dr Pepper \\
\end{longtable}

This pattern arises naturally from the NBD--Dirichlet model and is not a
sign of marketing failure.

\subsection{Duplication of Purchase}\label{duplication-of-purchase}

\nopagebreak

Brands ``share'' their buyers roughly in proportion to other brands'
market shares:

\begin{quote}
Buyers of Brand A are more likely to buy Brand B if Brand B is
\textbf{large}.
\end{quote}

Mathematically, the \textbf{Duplication of Purchase (DoP)} rate is
predictable from the Dirichlet component:

\[
\text{DoP}_{A,B} \approx s_B \left(\frac{1 - p_A}{1 - s_A}\right) 
\]

where\\
\(s_B\) = market share of brand B,\\
\(p_A\) = penetration of brand A,\\
\(s_A\) = market share of brand A.

\begin{center}\rule{0.5\linewidth}{0.5pt}\end{center}

\section{Estimating Parameters}\label{estimating-parameters}

\nopagebreak

Practical estimation steps:

\subsection{Step 1 --- Estimate Category
Parameters}\label{step-1-estimate-category-parameters}

From purchase or panel data:

Zero-Class Probability \(P(0)\):

\[
P(0) = \frac{\text{Number of non-buyers}}{\text{Total people}}
\]

Average category purchases:

\[
\mu = \text{Average category purchases per person (including zeros)}
\]

Then solve numerically for (\(k\)) in:

\[
P(0) = \left(\frac{k}{k+\mu}\right)^k 
\]

This can be done using \textbf{Goal Seek} or \textbf{Solver} in \emph{MS
Excel} or by numerical root-finding in \textbf{R}.

\begin{center}\rule{0.5\linewidth}{0.5pt}\end{center}

\subsection{Step 2 --- Estimate Brand
Parameters}\label{step-2-estimate-brand-parameters}

From brand-level data:

\begin{itemize}
\tightlist
\item
  Market shares (\(s_i\))
\item
  Number of brands (\(m\))
\end{itemize}

These feed into the Dirichlet component, which models how total category
purchases are divided across brands.

\subsubsection{Example: Computing Dirichlet Parameters from Market
Shares}\label{example-computing-dirichlet-parameters-from-market-shares}

Suppose a category has \textbf{four brands} with the following observed
market shares:

\begin{longtable}[]{@{}
  >{\raggedright\arraybackslash}p{(\linewidth - 6\tabcolsep) * \real{0.2361}}
  >{\raggedleft\arraybackslash}p{(\linewidth - 6\tabcolsep) * \real{0.2361}}
  >{\centering\arraybackslash}p{(\linewidth - 6\tabcolsep) * \real{0.2917}}
  >{\raggedleft\arraybackslash}p{(\linewidth - 6\tabcolsep) * \real{0.2361}}@{}}
\toprule\noalign{}
\begin{minipage}[b]{\linewidth}\raggedright
Brand
\end{minipage} & \begin{minipage}[b]{\linewidth}\raggedleft
Market Share \(s_i\)
\end{minipage} & \begin{minipage}[b]{\linewidth}\centering
Calculation \(\alpha_i = \alpha_0 s_i\)
\end{minipage} & \begin{minipage}[b]{\linewidth}\raggedleft
Dirichlet Parameter \(\alpha_i\)
\end{minipage} \\
\midrule\noalign{}
\endhead
\bottomrule\noalign{}
\endlastfoot
A & 0.40 & \(20 \times 0.40\) & 8.0 \\
B & 0.25 & \(20 \times 0.25\) & 5.0 \\
C & 0.20 & \(20 \times 0.20\) & 4.0 \\
D & 0.15 & \(20 \times 0.15\) & 3.0 \\
\textbf{Total} & \textbf{1.00} & & \(\alpha_0 = 20.0\) \\
\end{longtable}

\begin{center}\rule{0.5\linewidth}{0.5pt}\end{center}

\paragraph{Where does the value 20 come
from?}\label{where-does-the-value-20-come-from}

The constant \(\alpha_0\) is called the \textbf{concentration parameter}
(or sometimes the ``total precision'' or ``Dirichlet strength'').\\
It represents how \emph{tightly concentrated} the brand-choice
probabilities are around the observed market shares.

\begin{itemize}
\tightlist
\item
  A \textbf{larger} \(\alpha_0\) means \textbf{less variability} across
  consumers (everyone buys in proportion to market shares).\\
\item
  A \textbf{smaller} \(\alpha_0\) means \textbf{more heterogeneity}
  (some buyers specialise strongly in one brand).
\end{itemize}

The value of \(\alpha_0 = 20\) here is \textbf{not derived from the
shares themselves} --- it is either:

\begin{enumerate}
\def\labelenumi{\arabic{enumi}.}
\tightlist
\item
  \textbf{Chosen arbitrarily} for illustration (to make the numbers
  round), or\\
\item
  \textbf{Estimated from category data}, e.g., using observed
  duplication or repeat-buying rates.
\end{enumerate}

In practice, researchers often infer \(\alpha_0\) by fitting the
Dirichlet model to panel data.\\
For example, if the observed duplication of purchase (brand overlap) is
high, that implies a \textbf{larger} \(\alpha_0\); if duplication is low
(buyers are more brand-loyal), \(\alpha_0\) is smaller.

\begin{center}\rule{0.5\linewidth}{0.5pt}\end{center}

\subsubsection{Step 1 --- Compute individual
parameters}\label{step-1-compute-individual-parameters}

Given market shares \(s_i\) and chosen \(\alpha_0\):

\[
\alpha_i = \alpha_0 s_i, \qquad \sum_{i=1}^{4} \alpha_i = \alpha_0 = 20.
\]

\begin{center}\rule{0.5\linewidth}{0.5pt}\end{center}

\subsubsection{Step 2 --- Write the Dirichlet
PDF}\label{step-2-write-the-dirichlet-pdf}

\[
f(p_1, p_2, p_3, p_4)
= \frac{1}{B(8,5,4,3)} \;
p_1^{8-1} p_2^{5-1} p_3^{4-1} p_4^{3-1},
\qquad
\sum_i p_i = 1, \; p_i > 0.
\]

where the normalising constant \(B(8,5,4,3)\) is:

\[
B(8,5,4,3)
= \frac{\Gamma(8)\,\Gamma(5)\,\Gamma(4)\,\Gamma(3)}{\Gamma(8+5+4+3)}
= \frac{\Gamma(8)\,\Gamma(5)\,\Gamma(4)\,\Gamma(3)}{\Gamma(20)}.
\]

\begin{center}\rule{0.5\linewidth}{0.5pt}\end{center}

\subsubsection{Interpretation}\label{interpretation}

This fully specifies the Dirichlet distribution for brand-choice
probabilities in this four-brand market.\\
If we were to draw many samples from this distribution:

\begin{itemize}
\tightlist
\item
  The \textbf{mean} brand shares would remain \(s_i\),
\item
  But the \textbf{variance} and \textbf{degree of brand duplication}
  would depend on the total concentration \(\alpha_0\).
\end{itemize}

A smaller \(\alpha_0\) (say 5 instead of 20) would produce simulated
markets with \emph{more variability} in brand loyalty and purchase
overlap.

\subsubsection{\texorpdfstring{Distinguishing \(\alpha_0\) (Dirichlet)
from \(k\) (Negative
Binomial)}{Distinguishing \textbackslash alpha\_0 (Dirichlet) from k (Negative Binomial)}}\label{distinguishing-alpha_0-dirichlet-from-k-negative-binomial}

Although both \(\alpha_0\) and \(k\) describe heterogeneity, they belong
to \textbf{different parts} of the NBD--Dirichlet model and have
\textbf{distinct interpretations}:

\begin{longtable}[]{@{}
  >{\raggedright\arraybackslash}p{(\linewidth - 6\tabcolsep) * \real{0.2500}}
  >{\raggedright\arraybackslash}p{(\linewidth - 6\tabcolsep) * \real{0.2500}}
  >{\raggedright\arraybackslash}p{(\linewidth - 6\tabcolsep) * \real{0.2500}}
  >{\raggedright\arraybackslash}p{(\linewidth - 6\tabcolsep) * \real{0.2500}}@{}}
\toprule\noalign{}
\begin{minipage}[b]{\linewidth}\raggedright
Parameter
\end{minipage} & \begin{minipage}[b]{\linewidth}\raggedright
Model component
\end{minipage} & \begin{minipage}[b]{\linewidth}\raggedright
Describes variation in\ldots{}
\end{minipage} & \begin{minipage}[b]{\linewidth}\raggedright
Interpretation
\end{minipage} \\
\midrule\noalign{}
\endhead
\bottomrule\noalign{}
\endlastfoot
\(k\) & \textbf{Negative Binomial (NBD)} & How often different consumers
buy within the \emph{category} & A \textbf{small} \(k\) means high
variation in category purchase frequency --- some people buy often,
others rarely. \\
\(\alpha_0\) & \textbf{Dirichlet} & How consumers \emph{allocate} their
category purchases across brands & A \textbf{small} \(\alpha_0\) means
buyers are more brand-focused (less duplication); a \textbf{large}
\(\alpha_0\) means buyers spread purchases more evenly. \\
\end{longtable}

\begin{center}\rule{0.5\linewidth}{0.5pt}\end{center}

\paragraph{Conceptual summary}\label{conceptual-summary}

\begin{longtable}[]{@{}
  >{\raggedright\arraybackslash}p{(\linewidth - 6\tabcolsep) * \real{0.2500}}
  >{\raggedright\arraybackslash}p{(\linewidth - 6\tabcolsep) * \real{0.2500}}
  >{\raggedright\arraybackslash}p{(\linewidth - 6\tabcolsep) * \real{0.2500}}
  >{\raggedright\arraybackslash}p{(\linewidth - 6\tabcolsep) * \real{0.2500}}@{}}
\toprule\noalign{}
\begin{minipage}[b]{\linewidth}\raggedright
Model stage
\end{minipage} & \begin{minipage}[b]{\linewidth}\raggedright
Governs heterogeneity in
\end{minipage} & \begin{minipage}[b]{\linewidth}\raggedright
Key parameter
\end{minipage} & \begin{minipage}[b]{\linewidth}\raggedright
Example implication
\end{minipage} \\
\midrule\noalign{}
\endhead
\bottomrule\noalign{}
\endlastfoot
\textbf{Category purchase frequency} & how many category purchases
consumers make & \(k\) & Two consumers might make 2 vs 20 category
purchases in a period. \\
\textbf{Brand choice behaviour} & which brands those purchases go to &
\(\alpha_0\) & One consumer may buy almost only Brand A, another spreads
across all brands. \\
\end{longtable}

\begin{center}\rule{0.5\linewidth}{0.5pt}\end{center}

\paragraph{Quick intuition}\label{quick-intuition}

\begin{itemize}
\tightlist
\item
  \(k\) shapes the \textbf{vertical dimension} --- variation in
  \emph{how many category purchases} people make.\\
\item
  \(\alpha_0\) shapes the \textbf{horizontal dimension} --- variation in
  \emph{how those purchases are shared across brands}.
\end{itemize}

Together they define the market's structure:

\[
\text{NBD–Dirichlet:} \quad
\text{Frequency across consumers (via } k)
\;\;+\;\;
\text{Brand allocation across brands (via } \alpha_0).
\]

\begin{center}\rule{0.5\linewidth}{0.5pt}\end{center}

\paragraph{In practice}\label{in-practice}

When calibrating an NBD--Dirichlet model:

\begin{itemize}
\tightlist
\item
  \(k\) is typically estimated from \textbf{category-level repeat rates}
  or fitted to the category's purchase frequency distribution.\\
\item
  \(\alpha_0\) is estimated from \textbf{cross-brand duplication and
  loyalty} patterns.
\end{itemize}

They are \textbf{independent but complementary} --- one governs
\emph{how often} people buy, the other \emph{what mix} of brands they
buy.

\begin{center}\rule{0.5\linewidth}{0.5pt}\end{center}

\begin{center}\rule{0.5\linewidth}{0.5pt}\end{center}

\subsection{Typical Parameter Ranges}\label{typical-parameter-ranges}

\nopagebreak

\begin{longtable}[]{@{}
  >{\raggedright\arraybackslash}p{(\linewidth - 4\tabcolsep) * \real{0.2500}}
  >{\raggedright\arraybackslash}p{(\linewidth - 4\tabcolsep) * \real{0.2500}}
  >{\raggedright\arraybackslash}p{(\linewidth - 4\tabcolsep) * \real{0.5000}}@{}}
\toprule\noalign{}
\begin{minipage}[b]{\linewidth}\raggedright
Parameter
\end{minipage} & \begin{minipage}[b]{\linewidth}\raggedright
Typical Range
\end{minipage} & \begin{minipage}[b]{\linewidth}\raggedright
Interpretation
\end{minipage} \\
\midrule\noalign{}
\endhead
\bottomrule\noalign{}
\endlastfoot
\(\mu\) & 0.5 -- 2.0 & Mean number of category purchases per person \\
\(k\) & 0.3 -- 2.0 & Dispersion (heterogeneity) ~smaller → more
skewed \\
\(P(0)\) & 0.5 -- 0.9 & Proportion of category non-buyers \\
\(1 - P(0)\) & 0.1 -- 0.5 & Category penetration \\
\(s_i\) & 0.01 -- 0.40 & Market share of brand `i' \\
\end{longtable}

\begin{center}\rule{0.5\linewidth}{0.5pt}\end{center}

\section{Summary of Model Structure}\label{summary-of-model-structure}

\nopagebreak

\[
\begin{aligned}
P(X = x) & = \text{Negative Binomial for purchase incidence} \\
(p_1,\dots,p_m) & = \text{Dirichlet for brand choice} \\
\text{NBD–Dirichlet} & = \text{Combination predicting market structure}
\end{aligned}
\]

This framework explains:

\begin{itemize}
\tightlist
\item
  Regularities in brand performance,
\item
  Differences between large and small brands,
\item
  Stable duplication patterns across categories and markets.
\end{itemize}

\begin{center}\rule{0.5\linewidth}{0.5pt}\end{center}

\subsection{\texorpdfstring{Visualising \(k\) vs \(\alpha_0\): Two Kinds
of
Heterogeneity}{Visualising k vs \textbackslash alpha\_0: Two Kinds of Heterogeneity}}\label{visualising-k-vs-alpha_0-two-kinds-of-heterogeneity}

\subsubsection{Diagram 1 --- Where each parameter
acts}\label{diagram-1-where-each-parameter-acts}

\includegraphics[width=7.25in,height=4.92in]{NBD_Dirichlet_files/figure-latex/mermaid-figure-3.png}

\begin{center}\rule{0.5\linewidth}{0.5pt}\end{center}

\subsubsection{\texorpdfstring{Diagram 2 --- ``Two-axis'' intuition
(vertical = \(k\), horizontal =
\(\alpha_0\))}{Diagram 2 --- ``Two-axis'' intuition (vertical = k, horizontal = \textbackslash alpha\_0)}}\label{diagram-2-two-axis-intuition-vertical-k-horizontal-alpha_0}

\includegraphics[width=6.47in,height=6.5in]{NBD_Dirichlet_files/figure-latex/mermaid-figure-2.png}

\begin{center}\rule{0.5\linewidth}{0.5pt}\end{center}

\paragraph{Quick recap}\label{quick-recap}

\begin{longtable}[]{@{}
  >{\raggedright\arraybackslash}p{(\linewidth - 8\tabcolsep) * \real{0.2000}}
  >{\raggedright\arraybackslash}p{(\linewidth - 8\tabcolsep) * \real{0.2000}}
  >{\raggedright\arraybackslash}p{(\linewidth - 8\tabcolsep) * \real{0.2000}}
  >{\raggedright\arraybackslash}p{(\linewidth - 8\tabcolsep) * \real{0.2000}}
  >{\raggedright\arraybackslash}p{(\linewidth - 8\tabcolsep) * \real{0.2000}}@{}}
\toprule\noalign{}
\begin{minipage}[b]{\linewidth}\raggedright
Parameter
\end{minipage} & \begin{minipage}[b]{\linewidth}\raggedright
Layer
\end{minipage} & \begin{minipage}[b]{\linewidth}\raggedright
What varies
\end{minipage} & \begin{minipage}[b]{\linewidth}\raggedright
Lower value implies\ldots{}
\end{minipage} & \begin{minipage}[b]{\linewidth}\raggedright
Higher value implies\ldots{}
\end{minipage} \\
\midrule\noalign{}
\endhead
\bottomrule\noalign{}
\endlastfoot
\(k\) & NBD (category frequency) & How often people buy the
\textbf{category} & More dispersion (more light \& heavy buyers; higher
\(P(0)\)) & More homogeneity (buyers cluster around the mean) \\
\(\alpha_0\) & Dirichlet (brand allocation) & How purchases are split
across \textbf{brands} & More brand focus (lower duplication, higher
observed loyalty) & More even spread (higher duplication, similar
repertoires) \\
\end{longtable}

\textbf{Rule of thumb.}\\
Think \textbf{vertical} for \(k\) (how many category purchases) and
\textbf{horizontal} for \(\alpha_0\) (how those purchases are shared
across brands). They're \textbf{independent but complementary}.

\begin{center}\rule{0.5\linewidth}{0.5pt}\end{center}

\newpage

\section{Understanding ``Penetration'' in the Dirichlet
Model}\label{understanding-penetration-in-the-dirichlet-model}

The term \textbf{penetration} can mean two very different things in
marketing.\\
In the \textbf{Dirichlet model}, it refers to \textbf{buyer penetration}
--- not retail or distribution penetration.

\begin{center}\rule{0.5\linewidth}{0.5pt}\end{center}

\subsection{Buyer Penetration (Dirichlet
definition)}\label{buyer-penetration-dirichlet-definition}

In the NBD--Dirichlet framework, \emph{penetration} is:

\begin{quote}
\textbf{The proportion of all category buyers who purchased the brand at
least once during the period.}
\end{quote}

Formally:

\[
p_i = \frac{\text{Number of category buyers who bought brand } i}{\text{Total number of category buyers}}
\]

It describes \textbf{how many people bought}, not how many times they
bought, and not how many stores stock the brand.

\subsubsection{Example}\label{example}

If 10,000 people bought chocolate bars this year and\\
7,000 of them bought \emph{KitKat} at least once,\\
then KitKat's \textbf{penetration} = 70\%.

Penetration is a key driver in the Dirichlet system because:

\[
\text{Market share} \approx p_i \times \text{Average purchase rate (among buyers)}
\]

So, brands grow mainly by \textbf{gaining more buyers}, not by
dramatically increasing loyalty or purchase frequency.

\begin{center}\rule{0.5\linewidth}{0.5pt}\end{center}

\subsection{Retail or Numeric Penetration (Physical
Availability)}\label{retail-or-numeric-penetration-physical-availability}

In retail metrics, \emph{penetration} (also called \emph{numeric
distribution}) means something else:

\begin{quote}
\textbf{The proportion of retail outlets that stock the brand at least
once during a period.}
\end{quote}

\[
\text{Retail penetration} =
\frac{\text{Number of stores carrying the brand}}{\text{Total stores selling the category}}
\]

This is a \textbf{supply-side} measure of \emph{physical availability},
not a consumer behaviour measure.

\begin{center}\rule{0.5\linewidth}{0.5pt}\end{center}

\subsection{How the Two Meanings
Relate}\label{how-the-two-meanings-relate}

\begin{longtable}[]{@{}
  >{\raggedright\arraybackslash}p{(\linewidth - 6\tabcolsep) * \real{0.2500}}
  >{\raggedright\arraybackslash}p{(\linewidth - 6\tabcolsep) * \real{0.2500}}
  >{\raggedright\arraybackslash}p{(\linewidth - 6\tabcolsep) * \real{0.2500}}
  >{\raggedright\arraybackslash}p{(\linewidth - 6\tabcolsep) * \real{0.2500}}@{}}
\toprule\noalign{}
\begin{minipage}[b]{\linewidth}\raggedright
Concept
\end{minipage} & \begin{minipage}[b]{\linewidth}\raggedright
What it measures
\end{minipage} & \begin{minipage}[b]{\linewidth}\raggedright
Data source
\end{minipage} & \begin{minipage}[b]{\linewidth}\raggedright
Used in the Dirichlet model?
\end{minipage} \\
\midrule\noalign{}
\endhead
\bottomrule\noalign{}
\endlastfoot
\textbf{Buyer penetration} & \% of category buyers who purchased the
brand & Consumer panel or survey & ✅ Yes \\
\textbf{Retail penetration} & \% of outlets stocking the brand & Retail
audit or scanner data & ❌ No \\
\end{longtable}

Retail penetration enables buyer penetration --- a brand must be
available in stores before consumers can buy it --- but the Dirichlet
model itself only uses \textbf{buyer penetration} to explain
market-level patterns such as market share, duplication of purchase, and
loyalty.

\begin{center}\rule{0.5\linewidth}{0.5pt}\end{center}

\subsection{Summary}\label{summary}

\begin{longtable}[]{@{}
  >{\raggedright\arraybackslash}p{(\linewidth - 4\tabcolsep) * \real{0.3333}}
  >{\raggedright\arraybackslash}p{(\linewidth - 4\tabcolsep) * \real{0.3333}}
  >{\raggedright\arraybackslash}p{(\linewidth - 4\tabcolsep) * \real{0.3333}}@{}}
\toprule\noalign{}
\begin{minipage}[b]{\linewidth}\raggedright
Term
\end{minipage} & \begin{minipage}[b]{\linewidth}\raggedright
In plain English
\end{minipage} & \begin{minipage}[b]{\linewidth}\raggedright
In Dirichlet modelling
\end{minipage} \\
\midrule\noalign{}
\endhead
\bottomrule\noalign{}
\endlastfoot
\textbf{Buyer penetration} & \% of people who bought the brand & Core
input/output \\
\textbf{Retail penetration} & \% of stores that stock the brand &
External driver \\
\textbf{Market share} & \% of category volume sold by the brand &
Derived outcome \\
\end{longtable}

In short:

\begin{itemize}
\tightlist
\item
  \textbf{Retail penetration} reflects \emph{availability};\\
\item
  \textbf{buyer penetration} reflects \emph{actual buying}.
\end{itemize}

The Dirichlet model focuses on the latter --- the behaviour of people,
not stores.

\newpage

\section{Common Misunderstandings about the Dirichlet
Model}\label{common-misunderstandings-about-the-dirichlet-model}

Users sometimes confuse the Dirichlet model with psychological or
managerial theories of consumer behaviour.\\
The table below highlights some typical misconceptions and the correct
interpretations.

\begin{longtable}[]{@{}
  >{\raggedright\arraybackslash}p{(\linewidth - 2\tabcolsep) * \real{0.5000}}
  >{\raggedright\arraybackslash}p{(\linewidth - 2\tabcolsep) * \real{0.5000}}@{}}
\toprule\noalign{}
\begin{minipage}[b]{\linewidth}\raggedright
Misunderstanding
\end{minipage} & \begin{minipage}[b]{\linewidth}\raggedright
Clarification
\end{minipage} \\
\midrule\noalign{}
\endhead
\bottomrule\noalign{}
\endlastfoot
\textbf{1. ``Penetration means how many stores stock the brand.''} & In
the Dirichlet model, penetration refers to \textbf{buyers}, not stores
--- the \% of people who bought the brand at least once in a given
period. Retail distribution is a separate measure of \emph{physical
availability}. \\
\textbf{2. ``Loyalty drives market share.''} & It's the opposite:
\textbf{market share (penetration)} largely determines observed loyalty.
Larger brands have slightly higher repeat rates simply because they have
more buyers, not because their buyers are intrinsically more loyal. \\
\textbf{3. ``People are assumed to buy at random, so advertising doesn't
matter.''} & The model assumes statistical independence, not
\emph{psychological randomness}. Advertising, distinctive assets, and
availability all influence the \emph{probabilities} of brand choice ---
they shift the parameters (\(s_i\), \(p_i\)). \\
\textbf{4. ``If the laws always hold, marketing research is
pointless.''} & The laws describe \emph{expected patterns}, not tactical
details. Research helps diagnose \emph{deviations} (e.g., low
penetration vs share, weak duplication) and guides execution --- such as
which category entry points to strengthen. \\
\textbf{5. ``The Dirichlet predicts individual behaviour.''} & It
predicts \textbf{aggregate probabilities}, not who personally will
switch brands. It's a model of markets, not personalities. \\
\textbf{6. ``Small brands should focus on building loyalty before
reach.''} & In reality, almost all growth comes from \textbf{gaining
more buyers}. Loyalty rises automatically once penetration increases. \\
\textbf{7. ``Categories with subscriptions or infrequent purchases break
the model.''} & The same patterns emerge when analysed over an
appropriate time frame (e.g., contract renewal cycles, multi-year
purchases). Purchase opportunities differ, but the underlying
heterogeneity logic remains valid. \\
\textbf{8. ``Dirichlet models only apply to groceries.''} & The
empirical laws hold across \textbf{services, durables, B2B, and luxury}
--- anywhere people have repeat or substitute buying opportunities. \\
\end{longtable}

\begin{center}\rule{0.5\linewidth}{0.5pt}\end{center}

\subsection{Key Reminder}\label{key-reminder}

\begin{tcolorbox}[enhanced jigsaw, leftrule=.75mm, rightrule=.15mm, bottomtitle=1mm, opacityback=0, colbacktitle=quarto-callout-note-color!10!white, breakable, toprule=.15mm, title=\textcolor{quarto-callout-note-color}{\faInfo}\hspace{0.5em}{How to Think About the Dirichlet}, coltitle=black, colframe=quarto-callout-note-color-frame, toptitle=1mm, titlerule=0mm, arc=.35mm, bottomrule=.15mm, left=2mm, colback=white, opacitybacktitle=0.6]

The Dirichlet model doesn't explain \emph{why} people buy --- it
describes \emph{how} buyers collectively behave in steady-state markets.

Its purpose is descriptive, predictive, and diagnostic, not
psychological.

Use it to test whether brand data fit expected norms, and to identify
genuine marketing anomalies.

\end{tcolorbox}

\begin{center}\rule{0.5\linewidth}{0.5pt}\end{center}

\section{Modelling the Effects of Mental Availability in the NBD
Framework}\label{modelling-the-effects-of-mental-availability-in-the-nbd-framework}

An increase in \textbf{mental availability} --- for example, after an
effective advertising campaign --- can be represented within the
\textbf{Negative Binomial Distribution (NBD)} model as a change in a
brand's \emph{probability of being chosen}.

In the combined \textbf{NBD--Dirichlet system}, mental availability
affects the \textbf{Dirichlet layer} (how purchases are divided among
brands), but its observable effects appear through the \textbf{NBD
layer}, which predicts buyer penetration and purchase frequency.

\begin{center}\rule{0.5\linewidth}{0.5pt}\end{center}

\subsection{Where Mental Availability
Fits}\label{where-mental-availability-fits}

\begin{longtable}[]{@{}
  >{\raggedright\arraybackslash}p{(\linewidth - 6\tabcolsep) * \real{0.2500}}
  >{\raggedright\arraybackslash}p{(\linewidth - 6\tabcolsep) * \real{0.2500}}
  >{\raggedright\arraybackslash}p{(\linewidth - 6\tabcolsep) * \real{0.2500}}
  >{\raggedright\arraybackslash}p{(\linewidth - 6\tabcolsep) * \real{0.2500}}@{}}
\toprule\noalign{}
\begin{minipage}[b]{\linewidth}\raggedright
Model layer
\end{minipage} & \begin{minipage}[b]{\linewidth}\raggedright
Describes
\end{minipage} & \begin{minipage}[b]{\linewidth}\raggedright
Key parameters
\end{minipage} & \begin{minipage}[b]{\linewidth}\raggedright
Connection to marketing
\end{minipage} \\
\midrule\noalign{}
\endhead
\bottomrule\noalign{}
\endlastfoot
\textbf{NBD (Negative Binomial)} & Variation in category buying
frequency across consumers & \(\mu, k\) & Category buying patterns \\
\textbf{Dirichlet} & Allocation of those purchases across brands &
\(s_i\) (brand share of buying occasions) & Brand mental \& physical
availability \\
\end{longtable}

When advertising improves a brand's \emph{mental availability}, it
increases its probability of being chosen at each buying occasion ---
captured by \(s_i\).

\begin{center}\rule{0.5\linewidth}{0.5pt}\end{center}

\subsection{How the NBD Predicts
Penetration}\label{how-the-nbd-predicts-penetration}

In the NBD model, the probability of a consumer making \textbf{zero}
purchases of a given brand in the period is:

\[
P(0) = \left(\frac{k}{k + s_i \mu}\right)^k
\]

Therefore, the \textbf{buyer penetration} (probability of at least one
purchase) is:

\[
\text{Penetration} = 1 - P(0)
\]

An increase in \(s_i\) --- due to higher brand salience or mental
availability --- reduces \(P(0)\), meaning \textbf{more category buyers
now buy the brand}.

\begin{center}\rule{0.5\linewidth}{0.5pt}\end{center}

\subsection{What Information the Model
Needs}\label{what-information-the-model-needs}

\begin{longtable}[]{@{}
  >{\raggedright\arraybackslash}p{(\linewidth - 4\tabcolsep) * \real{0.3333}}
  >{\raggedright\arraybackslash}p{(\linewidth - 4\tabcolsep) * \real{0.3333}}
  >{\raggedright\arraybackslash}p{(\linewidth - 4\tabcolsep) * \real{0.3333}}@{}}
\toprule\noalign{}
\begin{minipage}[b]{\linewidth}\raggedright
Parameter
\end{minipage} & \begin{minipage}[b]{\linewidth}\raggedright
Description
\end{minipage} & \begin{minipage}[b]{\linewidth}\raggedright
Typical data source
\end{minipage} \\
\midrule\noalign{}
\endhead
\bottomrule\noalign{}
\endlastfoot
\(\mu\) & Average number of category purchases per buyer per period &
Panel data or market totals \\
\(k\) & Dispersion parameter (heterogeneity of buying rates) & Estimated
from category data \\
\(s_i\) & Brand's share of category purchase occasions & Market share or
assumed change \\
\(N\) & Total number of category buyers & Sets the base for buyer
counts \\
\(\Delta s_i\) & Change in share due to mental availability increase &
Estimated effect of advertising \\
\end{longtable}

\begin{center}\rule{0.5\linewidth}{0.5pt}\end{center}

\subsection{Example: Simulating a Mental Availability
Lift}\label{example-simulating-a-mental-availability-lift}

\begin{longtable}[]{@{}
  >{\raggedright\arraybackslash}p{(\linewidth - 6\tabcolsep) * \real{0.2500}}
  >{\raggedleft\arraybackslash}p{(\linewidth - 6\tabcolsep) * \real{0.2500}}
  >{\raggedleft\arraybackslash}p{(\linewidth - 6\tabcolsep) * \real{0.2500}}
  >{\raggedright\arraybackslash}p{(\linewidth - 6\tabcolsep) * \real{0.2500}}@{}}
\toprule\noalign{}
\begin{minipage}[b]{\linewidth}\raggedright
Parameter
\end{minipage} & \begin{minipage}[b]{\linewidth}\raggedleft
Before campaign
\end{minipage} & \begin{minipage}[b]{\linewidth}\raggedleft
After campaign
\end{minipage} & \begin{minipage}[b]{\linewidth}\raggedright
Comment
\end{minipage} \\
\midrule\noalign{}
\endhead
\bottomrule\noalign{}
\endlastfoot
Category mean purchases (μ) & 6.0 & 6.0 & Category constant \\
Dispersion (\(k\)) & 2.0 & 2.0 & Stable heterogeneity \\
Brand share (\(s_i\)) & 0.20 & 0.25 & Mental availability ↑ 25\% \\
\(P(0) = \left(\frac{k}{k + s_i \mu}\right)^k\) & 0.56 & 0.46 & Fewer
non-buyers \\
\textbf{Penetration (}\(1 – P(0)\)) & \textbf{0.44} & \textbf{0.54} &
Buyer base ↑ 10 points \\
Expected share & 20\% & 25\% & Matches assumed lift \\
\end{longtable}

An increase in \emph{mental availability} (here, a 25\% increase in
recall or consideration) translates into:

\begin{itemize}
\tightlist
\item
  noticeably higher \textbf{penetration}, and\\
\item
  a proportional lift in \textbf{market share} --- exactly as predicted
  by \emph{How Brands Grow}.
\end{itemize}

\begin{center}\rule{0.5\linewidth}{0.5pt}\end{center}

\subsection{Summary Interpretation}\label{summary-interpretation}

\begin{longtable}[]{@{}lll@{}}
\toprule\noalign{}
Concept & Direction of effect & Marketing meaning \\
\midrule\noalign{}
\endhead
\bottomrule\noalign{}
\endlastfoot
↑ Mental availability & ↑ \(s_i\) & More likely to be chosen \\
↑ \(s_i\) & ↓ \(P(0)\) & Fewer non-buyers → higher penetration \\
↑ Penetration & ↑ Market share & Growth through more buyers \\
\end{longtable}

In short: \textbf{Advertising and distinctive assets expand mental
availability}, which mathematically increases a brand's choice
probability \(s_i\).

Through the NBD, this produces measurable gains in buyer penetration and
market share, even when the category's total purchasing rate (μ) and
heterogeneity (k) remain constant.

\begin{center}\rule{0.5\linewidth}{0.5pt}\end{center}

\newpage

\section*{Appendix}\label{appendix}
\addcontentsline{toc}{section}{Appendix}

Technical explanation that no-one is expected to read.

\subsection*{The Dirichlet--Multinomial
Link}\label{the-dirichletmultinomial-link}
\addcontentsline{toc}{subsection}{The Dirichlet--Multinomial Link}

The Dirichlet is a prior on brand choice probabilities.\\
If each shopper has her own probability vector

\[
(p_1, p_2, \dots, p_m) \sim \text{Dirichlet}(\alpha_1, \alpha_2, \dots, \alpha_m),
\]

then conditional on those probabilities, the shopper's actual purchase
\emph{realizations} across brands in a period follow a
\textbf{multinomial} distribution.

That is, if the shopper makes \(x\) total purchases in the category in
the period, and we observe counts for each of \(m\) brands
\((x_1, x_2, \dots, x_m)\) with \(\sum_{i=1}^m x_i = x\), then:

\[
P(x_1, \dots, x_m \mid p_1, \dots, p_m, x)
= \frac{x!}{\prod_{i=1}^m x_i!}
\prod_{i=1}^m p_i^{x_i}.
\]

Its \textbf{probability density function (PDF)} is:

\$\$ f(p\_1, \dots, p\_m;, \alpha\_1, \dots, \alpha\emph{m) =
\frac{1}{B(\alpha_1, \dots, \alpha_m)} \prod}\{i=1\}\^{}m
p\_i\^{}\{\alpha\_i - 1\}

\textbackslash{}

\text{where } \quad \sum\_\{i=1\}\^{}m p\_i = 1, \quad  p\_i
\textgreater{} 0 \$\$

and the \textbf{normalising constant} \(B(\cdot)\) (the multivariate
Beta function) is:

\[
B(\alpha_1, \dots, \alpha_m)
= \frac{\prod_{i=1}^m \Gamma(\alpha_i)}
{\Gamma\!\left(\sum_{i=1}^m \alpha_i\right)}.
\]

where:

\begin{itemize}
\tightlist
\item
  \(p_i\) = Proportion of purchases allocated to brand \(i\)
\item
  \(\alpha_i\) = Dirichlet shape parameter for brand \(i\)
\item
  \(B(\cdot)\) = Multivariate \emph{Beta} function ensuring the
  distribution integrates to 1
\end{itemize}

Now we \textbf{integrate out} the unobserved personal probabilities
\((p_1,\dots,p_m)\) by using the Dirichlet prior.\\
This yields the \textbf{Dirichlet--multinomial} distribution:

\[
P(x_1, \dots, x_m \mid \alpha_1, \dots, \alpha_m, x)
=
\frac{x!}{\prod_{i=1}^m x_i!}
\,
\frac{\Gamma\!\left(\sum_{i=1}^m \alpha_i\right)}
     {\Gamma\!\left(x + \sum_{i=1}^m \alpha_i\right)}
\prod_{i=1}^m
\frac{\Gamma(x_i + \alpha_i)}{\Gamma(\alpha_i)}.
\]

Key points:

\begin{itemize}
\tightlist
\item
  (\(x_i\)) = number of purchases of brand `i' made by this person in
  the period\\
\item
  (\(x = \sum_i x_i\)) = total category purchases for this person in the
  period\\
\item
  (\(\alpha_i\)) = Dirichlet shape parameters, typically proportional to
  long-run brand shares\\
\item
  (\(\Gamma(\cdot)\)) = gamma function
\end{itemize}

This distribution gives the probability of observing a particular
``basket split'' across brands, for a given total number of purchases
\(x\), \textbf{without needing to know that shopper's individual
loyalties explicitly}. The \(\alpha_i\) encode typical brand
attractiveness / repertoire weights at the population level.

\begin{center}\rule{0.5\linewidth}{0.5pt}\end{center}

\subsection*{Connecting to the
NBD--Dirichlet}\label{connecting-to-the-nbddirichlet}
\addcontentsline{toc}{subsection}{Connecting to the NBD--Dirichlet}

We now have two layers:

\begin{enumerate}
\def\labelenumi{\arabic{enumi}.}
\item
  \textbf{How many category purchases are made?}\\
  The total category purchase count \(X\) for a shopper follows a
  Negative Binomial with parameters \((\mu, k)\):

  \[
  P(X = x) =
  \frac{\Gamma(x + k)}{\Gamma(k)\, x!}
  \left(\frac{k}{k + \mu}\right)^k
  \left(\frac{\mu}{k + \mu}\right)^x,
  \quad x = 0,1,2,\ldots
  \]
\item
  \textbf{How are those} \(x\) purchases allocated across brands?\\
  Conditional on \(X = x\), the vector of brand-level counts
  \((x_1, \dots, x_m)\) with \(\sum_i x_i = x\) follows a
  Dirichlet--multinomial with parameters
  \((\alpha_1, \dots, \alpha_m)\):

  \[
  P(x_1, \dots, x_m \mid X = x)
  =
  \frac{x!}{\prod_{i=1}^m x_i!}
  \,
  \frac{\Gamma\!\left(\sum_{i=1}^m \alpha_i\right)}
       {\Gamma\!\left(x + \sum_{i=1}^m \alpha_i\right)}
  \prod_{i=1}^m
  \frac{\Gamma(x_i + \alpha_i)}{\Gamma(\alpha_i)}.
  \]
\end{enumerate}

The full \textbf{NBD--Dirichlet model} is the hierarchical combination:

\begin{itemize}
\tightlist
\item
  First layer (incidence): draw \(X \sim \text{NegBin}(\mu, k)\).
\item
  Second layer (allocation): draw
  \((x_1,\dots,x_m) \mid X \sim \text{Dirichlet–Multinomial}(\alpha_1,\dots,\alpha_m).\)
\end{itemize}

This two-stage structure produces the classic \emph{Ehrenberg--Bass}
predictions:

\begin{itemize}
\tightlist
\item
  \textbf{Penetration} of each brand
\item
  \textbf{Average purchase frequency} per brand's buyers
\item
  \textbf{Duplication of purchase} between brands
\item
  The \textbf{Double Jeopardy} pattern\\
  (small brands have fewer buyers and slightly less loyalty)
\end{itemize}

\begin{center}\rule{0.5\linewidth}{0.5pt}\end{center}

\subsection*{Why this matters
(intuitively)}\label{why-this-matters-intuitively}
\addcontentsline{toc}{subsection}{Why this matters (intuitively)}

\begin{itemize}
\item
  The Negative Binomial captures:\\
  \emph{``How often do people buy in this category at all?''}
\item
  The Dirichlet--multinomial captures:\\
  \emph{``Given that they're buying, how do they split those purchases
  across brands?''}
\end{itemize}

That's the full engine under \emph{How Brands Grow}.

\newpage

\section*{References}\label{references}
\addcontentsline{toc}{section}{References}

\nopagebreak

\begin{itemize}
\tightlist
\item
  Sharp, B. (2010). \emph{How Brands Grow: What Marketers Don't Know}.
  Oxford University Press. ISBN 9780195573565
  https://doi.org/10.1093/oso/9780195573565.001.0001\\
\item
  Sharp, B., Kennedy, R., \& Sharp, A. (2019). \emph{How Brands Grow,
  Part 2: Emerging Markets, Services, Durables, New and Luxury Brands}.
  Oxford University Press. ISBN 9780195596267
  https://doi.org/10.1093/oso/9780195596267.001.0001\\
\item
  Sharp, B., Wright, M., \& Goodhardt, G. J. (2002). Purchase loyalty is
  polarised, how brand loyalty differs by customer size.
  \emph{Australasian Marketing Journal (AMJ)}, 10(3), 7--20.
  https://doi.org/10.1016/S1441-3582(02)70138-5\\
\item
  Ehrenberg, A. S. C., Uncles, M. D., \& Goodhardt, G. J. (2004).
  Understanding brand performance measures: Using Dirichlet
  distributions. \emph{Journal of Business Research}, 57(12),
  1307--1325. https://doi.org/10.1016/S0148-2963(03)00041-1\\
\item
  Goodhardt, G. J., Ehrenberg, A. S. C., \& Chatfield, C. (1984).\\
  The Dirichlet: A comprehensive model of buying behaviour.\\
  \emph{Journal of the Royal Statistical Society. Series A (General)},
  147(5), 621--655.\\
  https://doi.org/10.2307/2981578
\end{itemize}




\end{document}
